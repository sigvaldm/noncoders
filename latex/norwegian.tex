\documentclass[a4paper]{report}
\usepackage[utf8]{inputenc}
\usepackage[T1]{fontenc}
\usepackage[norsk]{babel}

%\pretolerance = 2000 % Badness tolerated before starting to hyphenate
%\tolerance = 4000 % Badness tolerated after hyphenation
%\hbadness = 4000 % Maximum badness before giving warning
%\hyphenpenalty = 100 % Penalty in badness for inserting a hyphen

\begin{document}

% How to print these variables
\the\pretolerance

\the\tolerance

\the\hbadness

\the\hyphenpenalty

\begin{minipage}{0.5\textwidth}
Da jeg kom til Brekke-sagen, var himmelen overskyet; det var alt temmelig mørkt, bare ved den nordvestlige rand av synskretsen sto en eplegrønn strime, som kastet et dempet lys på sagdammens stille flate. Jeg gikk ut på lensen og gjorde noen kast, men med lite hell. Ikke et pust rørte seg; vinden syntes å ha gått til hvile, og bare fluene mine brakte det blanke vannet til å skjelve.

Neste morgen sto risen fælt tidlig opp og strøk til skogs, og aldri før var han av gårde, så tok Askeladden og kongsdatteren på å lete under dørhellen etter hjertet hans; men alt det de grov og lette, så fant de ikke noe. "Denne gangen har han lurt oss," sa prinsessen, "men vi får vel prøve ham enda en gang." Så sanket hun alle de vakreste blomster hun kunne finne, og strødde rundt om dørhellen - den hadde de lagt slik den skulle ligge; og da det led mot den tiden de ventet risen hjem, krøp Askeladden under sengen igjen.

Men best som det var, stakk den katta som hadde vøri i ferd med å velte gryta, labben sin innafor ringen, jussom hu hadde hug til å få tak i skreddern. Men da skreddern såg det, løyste 'n på telgjekniven og heldt 'n ferdig. Rett som det var, slo katta labben innafor ringen att, men i samme blinken hakka skreddern labben ta, og alle kattene ut det forteste dom vant, med ul og med skrik.

I det samme så de trollene komme settende, og de var så store og digre at hodene på dem var jevnhøye med furutoppene. Men de hadde bare ett øye sammen alle tre, og det skiftedes de til å bruke; de hadde et hull i pannen, som de la det i, og styrte det med hånden; den som gikk foran, han måtte ha det, og de andre gikk etter og holdt seg i den første.




De andre sa hun skulle akte seg, men hun ble ved sitt, og da klokken vel kunne være litt over ett, sto hun opp og la under bryggekjelen og hadde på ròsten. Men hvert øyeblikk sloknet det under kjelen, og det var liksom én kastet brannen ut over skorstenen, men hvem det var, kunne hun ikke se. Hun tok og samlet brannene den ene gangen efter den andre, men det gikk ikke bedre, og ròsten ville heller ikke gå. Til sist ble hun kjed av dette, tok en brann og løp med både høyt og lavt, og svingte den og ropte:

Men Askeladden ville og skulle avsted, og han tagg og ba så lenge til kongen måtte la ham reise. Nå hadde kongen ikke annet enn en gammel fillehest å la ham få, for de seks andre kongssønnene og følget deres hadde fått alle de andre hestene han hadde; men det brydde ikke Askeladden seg om; han satte seg opp på den gamle skabbete hesten, han. ``Farvel, far!'' sa han til kongen; ``jeg skal nok komme igjen, og kanskje jeg skal ha med meg brødrene mine også,'' og dermed reiste han.

I det samme så de trollene komme settende, og de var så store og digre at hodene på dem var jevnhøye med furutoppene. Men de hadde bare ett øye sammen alle tre, og det skiftedes de til å bruke; de hadde et hull i pannen, som de la det i, og styrte det med hånden; den som gikk foran, han måtte ha det, og de andre gikk etter og holdt seg i den første.

Da pannekaken hørte dette, ble den redd, og rett som det var, så vendte den seg av seg selv og ville ut av pannen; men den falt ned igjen på den andre siden, og da den hadde stekt seg litt på den også, så den ble fastere i fisken, spratt den ut på gulvet og trillet avsted som et hjul ut gjennom døren og bortetter veien.

Utpå natten satte prinsen en ring på fingeren til Åse, og den var så trang at hun ikke kunne få den av seg igjen; for prinsen kunne nok skjønne at det ikke gikk riktig til, og så ville han ha et merke han kunne kjenne igjen den på som var den rette. Da prinsen hadde sovnet, kom prinsessen og jaget Åse ned i gåsestien igjen, og la seg selv i rommet hennes.

\end{minipage}


\end{document}