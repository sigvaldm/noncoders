\chapter{Introduction}
Programmers, engineers and scientists as well as computer hobbyists have had the opportunity to learn a variety of different computer technologies which facilitates their everyday work in front of the computer. They have acquired a ``toolbox'' of knowledge about various technologies, in particular programming languages, and this toolbox enables them to do things more efficiently and professionally and often with more ease. In comparison, people who neither have jobs related to computers nor a special interest in them often end up doing things inefficiently and impractically.

Consider for instance that you have hundreds of photos and you’d like to change the resolution on all of them. Or maybe add a watermark before sharing them? Or change the filename to include the date? Such tasks are tedious if you have to apply the changes to one photo at the time. However, the file names of all files can be changed using a single command in the Bash command line interface. A script that adds a watermark to every image takes less then 20 lines of code using the Python scripting language. Of course, programming takes time to learn, especially if you want to learn it well enough to become a programmer by trade. But with the emergence of modern low-threshold scripting languages like Python, the effort to learn \emph{some} programming is within reach for the majority. Why, then, should such handy techniques still be reserved for geeks only? And for sure, knowledge about technology among laymen will have to increase in the future.

Another thing many may benefit from is learning to write documents using a professional typesetting tool like \LaTeX, be it job applications, CV’s, reports, slideshow presentations, etc. Because let’s admit it: Most of us, geeks included, aren’t typographists. We don’t know exactly which font to use, which size, which line spacing and which margins to use everywhere to make a document look professional. With \LaTeX{}, you focus on what stuf \emph{is} and \LaTeX{} takes care of how stuff \emph{looks}. And the way you tell \LaTeX{} what stuff \emph{is} is by code. It doesn't have to be hard, for instance, to start a new chapter called ``Introduction'', you would write:

\latexone{\chapter{Introduction}}

As a side-effect, since \LaTeX{} now knows the name of all your chapters, it can easily auto-generate a table of contents where you insert this very simple command:

\latexone{\tableofcontents}

As a last example, have you ever had to collaborate with someone to write a document in a standard text processing tool like Microsoft Word or LibreOffice Writer? Then you’ll know it’s a mess. You have to pass around the document, and make sure no two people edit the same document at once. Despite working on completely different parts of the document. Documents written in \LaTeX{}, as well as programs written in languages such as Python, can be efficiently shared among multiple users using a Version Control System (VCS) such as Git. Moreover, everyone can work on it simultaneously without stepping on each other's toes! VCSs keeps track of the history of the documents, so if you regret your latest changes to the document, you can easily rewind using Git.

The purpose of this little book is to provide a small, but versatile toolbox for the majority of people which do not have a computer oriented job or hobby. The tools presented are carefully selected because together, they form a minimal toolbox allowing the reader to use a computer in such a rich way. Further on, the presentation will not focus on graphical programs but on computer languages, given that the present generation grew up with technology and are mostly capable of figuring out of graphical programs themselves. We will study the command line, the Python programming language, the \LaTeX{} typesetting tool, and finally the Git version control system. Admittedly, we will only scratch the surface of these tools, not even going to the same depth as many other beginner's books. Otherwise it wouldn't be a minimal toolbox anymore, and by no means a ``minimal'' book. However, the aim is that by the end of this book, you should know some basics and be self-going enough to search for further knowledge as you need to.

The chapters are self-contained and may be read independently of each other (although it is highly advantageous to know some computer language before learning Git).